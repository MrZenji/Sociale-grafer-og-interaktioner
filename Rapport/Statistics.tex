Now that we have data to classify and the classifier, we are able to classify each profile,
on how probable the person will follow back. We do this by going through the list of people we have gathered, then we set up 9 probabilities going from 55\% to 95\%. We do not care about probabilities under 55\%, because we want the classifier to make better qualified guesses than a normal human being.\\ We then store the number of people the classifier have guesses, as who would follow back(good), and who would not(bad). Then we check all of the profiles on how many followed back, to see how many times the classifier made a wrong guess. In the end we summarize how many times the classifier made a right and wrong guess.\\
 
\begin{tabular}{ c | c | c | c | c |}
\cline{2-5}
& \multicolumn{2}{ |c| }{Good} & \multicolumn{2}{ |c| }{Bad}\\ \hline
Probability & Guess & Wrong & Guess & Wrong\\ \hline
95\% & 4 & 3 & 954 & 108 \\ \hline
90\% & 14 & 11 & 944 & 106 \\ \hline
85\% & 48 & 42 & 910 & 103 \\ \hline
80\% & 78 & 67 & 880 & 98 \\ \hline
75\% & 129 & 112 & 829 & 92 \\ \hline
70\% & 208 & 179 & 750 & 80 \\ \hline
65\% & 267 & 231 & 691 & 73 \\ \hline
60\% & 327 & 284 & 631 & 66 \\ \hline
55\% & 385 & 337 & 573 & 61 \\ 
\hline
\end{tabular}

We then divide the number of wrong with the guess, for both good and bad.\\

\begin{tabular}{ c | c | c |}
&\multicolumn{2}{| c |}{\% of wrong}\\ \hline
\multicolumn{1}{c |}{Probability} & Good & Bad\\ \hline
\multicolumn{1}{c |}{95\%} & 75 & 11.3\\ \hline
\multicolumn{1}{c |}{90\%} & 78.6 & 11.2\\ \hline
\multicolumn{1}{c |}{85\%} & 87.5 & 11.3\\ \hline
\multicolumn{1}{c |}{80\%} & 85.9 & 11.1\\ \hline
\multicolumn{1}{c |}{75\%} & 86.8 & 11.1\\ \hline
\multicolumn{1}{c |}{70\%} & 86.1 & 10.7\\ \hline
\multicolumn{1}{c |}{65\%} & 86.9 & 10.6\\ \hline
\multicolumn{1}{c |}{60\%} & 86.7 & 10.5\\ \hline
\multicolumn{1}{c |}{55\%} & 87.5 & 10.6\\ \hline
\end{tabular}

Some other numbers we are able to look at are how precise the classifier are, this is done by creating a test set.
After we create the test set we use \textit{nltk.classify.accuracy(classifier, test\_set)} to get the accuracy of the classifier. But the data we normally only use for the classifier, will be divide so that 70\% go to the training set for the classifier and 30\% goes to the test set. We run this 100 times, with the data being shuffled each time and take the average in the end. This tells us that the classifier is 51.603\% accurate, which is not a lot, but still 1.5\% better than a normal guess.