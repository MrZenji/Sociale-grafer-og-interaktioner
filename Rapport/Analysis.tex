Before we can talk about which probability, with our dataset, is the most efficient one, we need to look at which is worst, getting a wrong with those we thought was good or getting a wrong with those we thought was bad. Getting a wrong when we think they are good, means that for each wrong out the 200 persons we follow a day, will not follow us back, and therefore we will be getting less people following us a day. Getting a wrong when we think they are bad, means we wont follow them, even though they would follow us back, this number needs to be as low as possible, because these would be a potential follower lost forever.\\
\\
This is why it is more important to have a low wrong in bad, than a low wrong in good. The percentage of wrong in good is rising when we lower the probability, this makes a lot of sense, because we allow more risky judgement. But lowering the probability also means a lowering of the percentage of wrong in bad which is a good thing. The lowest percentage of wrong in bad is when the classifier is at a 60\% probability. Also at this classifier probability the wrongs in good is lower then the  65,75 and 85\%. So with our dataset of 254 people, the 60\% classifier gives the lowest loss of people who would follow us, but also lets 86.7\% wrong though the filter.\\
\\
The reason we look at how many times this classifier makes a wrong is because of its low accuracy, that was 51.6\%. But if we looked at how many times it made it right, then we would have to only follow those that have a 90\% probability, because there is a 25\% change of being right, and its the highest of them all. But this would only be 1 person out of 958, so we would need to run the daily routine 5 days to get 1 follower.