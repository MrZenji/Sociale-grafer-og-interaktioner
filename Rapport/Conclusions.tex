Throughout the course of this project, we have created a Twitter bot, using the Twitter REST API. With this bot we were able to automatic follow people living in San Francisco, and create original tweets fitting our bots personality. We also retweet'et tweets, that had specific keyword in them. Because Twitter doesn't allow you to follow an infinite amount of people a day, we created a classifier to specify if the people we found would have a high probability to follow us back. Due to the low accuracy of the classifier on the people who will follow us back, we would still follow people that doesn't follow us back, and we would ignore some that had the potential to follow us. However we have a very high accuracy of close to 85\% on the people we know, don't follow us back. That said we can use our classifier to sort out most of the twitter profiles, who wouldn't follow us back. This will be useful due to twitters limit on how many persons you can follow per day, before you are going to be suspected of being a bot and banned.\\
On the question of, if humans are susceptible to follow social bots, we can conclude that we only got 127 followers out of 1235, which corresponds that we have a 10\% chance to gain a follow, if the profile is a real human. This is actually pretty decent, when thinking about the number of people existing in the twitter network. Therefore we can say, that people are susceptible to social bots.
A thought we had was when we followed people, that we sent them a private message, that might have given them more encouragement to follow us back\\
To other bot designers we would say that, it is great idea to create a profile with interest are general and appears appealing to people.